
\subsection{Endnotes}

According to Rathgeber's notes, the error effect was discovered in 1974. Rathgeber's papers glossed
over some matters that caused considerable confusion and doubt about the value of his work. I have
tried to make explicit those sources of ambiguity. Rathgeber discusses the need to separate out the
``functions of money''. This was a source of confusion and lack of precision, which I tried to deal
with using the notion of transactions as a way to delineate the different functions of money.
Rathgeber did not make a clear distinction between a currency as a technical property as compared
to the market behaviour of people, but the notion was implicit in his work.  This was also a source
a confusion, because people, quite reasonably, object to the application of engineering method
directly to social interaction. Making explicit the difference between markets and currency
hopefully clear that engineering method is applied to currencies, not social interaction. Rathgeber
was working in the context of a single, national, centralized currency. I have extended the
application of his ideas to digital currencies.  At the time he wrote his private papers, he had
access only to Australian inflation and unemployment data. I have extended this to data available up
to the time of publication.






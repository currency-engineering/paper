\chapter{Exchange Transactions}

\subsection{Markets}

One effort to explain the sustained failure or markets to equilibriate at the aggregate level is to
try to explain failure of equilibriation as a result of the way individual economic behaviour
aggregates to failure or otherwise of markets for single products to equilibriate, and further to
explain failure of markets to equilibriate at the aggregate level.

A fundamental method of science and engineering is to assume as a first step, is to use the mean
value to aggregate a collection of micro-level behaviours. Often this turns out not to be correct,
but invariably, in virtually every system we seek to explain, there are some parts of the system we
explain away by averaging out noisy behaviour. 

If we use the same technique for understanding economic behaviour, we would, as a first step assume
that we can average markets for single goods or services, result in an aggregate supply or demand
close to zero.

If we use the same technique for understanding economic behaviour, we would, as a first step assume
that we can aggregate our model of supply and demand for single goods or services, and arrive at a
aggregate where aggregate supply or demand is close to zero.

Since this conclusion is contrary to facts, economists have directed their efforts at modifying the
supply and demand model in many ways in an effort to explain this contradiction between fact and
theory.

What is clear, however, is that the explanation has to be sufficiently fundamental to explain the
remarkably consistent fact of excess aggregate supply and the rarity of aggregate market
equilibrium. As put forward by Lucas, economists have yet to find a convincing understanding of
this fact, let alone to find a solution to the problem of equilibrium failure or the problem of
a sustained positive unemployment rate. 

Since our explanation of these facts is outside is not a part of the supply and demand model, we
assume our simplest model of market behaviour, and that markets at the aggregate level do in fact
equilibriate, relying on the law of large averages.


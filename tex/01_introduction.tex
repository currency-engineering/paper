% hello.tex - Our first LaTeX example!
\documentclass{article}
\begin{document}
The core model by which we understand economic processes is the interaction of and supply and
demand. This model has been considerably succesful in understanding why markets tend towards
equilibrium equality of supply and demand, but has failed to explain the remarkable persistence of
an excess of aggregate supply as indicated by a sustained and positive rate of unemployment.

In 1777 David Hume observed, despite theoretical considerations to the contrary, that the
state of a nation's currency had observable affects on the state of the economy,

\begin{quotation}
If we consider any one kingdom by itself, it is evident, that the greater or less plenty of money
is of no consequence; since the prices of commodities are always proportioned to the plenty of
money, and a crown in Harry VII.’s time served the same purpose as a pound does at present ... It is
indeed evident, that money is nothing but the representation of labour and commodities, and serves
only as a method of rating or estimating them. Where coin is in greater plenty; as a greater
quantity of it is required to represent the same quantity of goods; it can have no effect, either
good or bad, taking a nation within itself; any more than it would make an alteration on a
merchant’s books, if, instead of the Arabian method of notation, which requires few characters, he
should make use of the Roman, which requires a great many ... **But notwithstanding this
conclusion**, which must be allowed just, it is certain, that, since the discovery of the mines in
America, industry has encreased in all the nations of Europe, except in the possessors of those
mines; and this may justly be ascribed, amongst other reasons, to the encrease of gold and silver.
Accordingly we find, that, in every kingdom, into which money begins to flow in greater abundance
than formerly, every thing takes a new face: labour and industry gain life; the merchant becomes
more enterprising, the manufacturer more diligent and skilful, and even the farmer follows his
plough with greater alacrity and attention. This is not easily to be accounted for, if we consider
only the influence which a greater abundance of coin has in the kingdom itself, by heightening the
price of commodities, and obliging every one to pay a greater number of these little yellow or white
pieces for every thing he purchases.
\end{quotation}

A simple model that aggregates equilibria in each market implies an equilibrium at the macro-level.
This model suggests macro-economic conditions as shown in Figure 1.

% <p class="figure_title">Figure 1.2 Supply and Demand Model Prediction</p>
% <div id="fig_supply_and_demand_prediction"></div>
% 
% However, contrary to Figure 1. which implies full-employment Figure 2. shows a vastly differing
% results.
% 
% <p class="figure_title">Figure 1.3 Inflation Rate vs. Unemployment Data</p>
% <div id="fig_ui_all_data"></div>
% 
% <p class="figure_title">Figure 6.1 Unemployment vs. Inflation Rate</p>
% <div id="fig_ui"></div>

\end{document}

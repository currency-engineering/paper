\section{Exchange Transactions and Errors}
\label{section:exchange_transactions_and_errors}

% A back-of-an-envelope calculation reveals that within an order of magnitude this hypothesis results
% in all the variables reacting with each other within an orderin a way we have observed in economic
% systems. An error rate of 5 percent is maybe reasonable. An error rate of 10 percent seems
% unreasonably large. From personal experience visiting a supermarket, there is some chance that a
% line of goods we were expecting to buy is not available. This is probably less than 1 in every 10
% purchases we are expecting to make. On the other hand 1 percent probably seems too small, though not
% impossible. Communication systems typically require 2.5 - 10 times the error rate to expend to
% compensate for errors, so we could assume a reasonable number is 5. The hypothesis implies that an
% increase in the growth rate  1 percent increase in the growth rate will cause a 1 in 5 percent
% increase the unemployment rate for a given inflation rate. This is certainly in the right direction
% but seems a little small. Unemployment rates typically fluctuate between 3 and 10 percent, but to
% drive a reduction in the unemployment rate from 10 percent to 3 percent would would require 35
% percent increase in the growth rate (with fixed inflation rate), which seems excessive. Nevertheless
% there is no variable that is reacting in an unexpected direction.
% 
% Given our rough numbers above, an increase of 5 percent in the inflation rate with a fixed growth
% rate, would result in a 1 percent reduction in the unemployment rate. This is certainly in line with
% the two most prominent economic papers on the relationship between the inflation rate and the
% unemployment rate, firstly Phillips and secondly Lucas.
% 
% TODO: Phillips figure and Lucas figure 

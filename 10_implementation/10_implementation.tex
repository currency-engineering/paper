\section{Implementation}
\label{sec:implementation}

\subsection{Demand Indexation}

The demand index is a value published by a monetary authority or an algorithm. It directly adjusts
the total money supply by indexing all account values. Accounts exist as a base value. The values
appearing in called face values. The face value is a multiple of the base value and the demand
index. Prices are denominated in this value, and base values are generally hidden from the user.
Base values are transfered between accounts.

\subsection{Exchange Transactions}

Valid exchange transactions require a record of the payment, the quantity of goods to be
transactions, and the goods category. It is always possible for buyers and sellers to collude to
write in incorrect goods categories. The importance of recording the goods category is for measuring
aggregate properties of a currency's transactions, and also to record that the transaction was
indeed an exchange transaction and that no future contractural obligations remain. By using exchange
transactions buyers and seller forfeit all future financial obligations, thereby reducing the
utility of using exchange transactions for facilitating contract transactions.

\subsection{Data Record}

A record of transaction payment, goods category and quantity of goods are recorded, and used to
calculate the $Q$, $P$ and $F$. The price index $P$ is published and used for calculation of time
transactions. Methods to protect the privacy of accounts must be used. 

\subsection{Indexed Unit of Account}

\subsection{Time Transactions}

Time transactions involve an initial payment and a later repayment. All monetary values in contracts
for time transactions are written in indexed units of account and repayments are not facilitated on
contracts in other units. The indexed unit of account records a base value which is then indexed by
the price index. The receiving account for repayments for time transactions must be the same account
from which the initial payment is made.

\subsection{Contract Transactions}

Contract transactions are prevented by only allowing repayments of time transactions into the same
account from which the initial payment is made. Transactions that are not recorded as time
transactions are not acceptable for setting up financial contractural obligations, so no party has
any financial legal rights beyond time transactions. There remains likelihood that other
transactions are used to achieve contract transactions, but the important requirement is that
contract transactions are sufficiently limited to prevent effects that disturb the economy at
aggregate level through interest rates, disturbance of control of aggregate demand, or positive
feedback in speculative activity. 

\subsection{Control}

The demand index is determined using the data record so as to regulate $\Delta F / F$. Occassional
unemployment surveys may be taken to confirm that the set point for $\Delta F / F$ is sufficiently
high.

\subsection{Intermediate Accounts}

Because exchange transactions require a record of goods category, there is a need for intermediate
accounts that sit between accounts that make exchange transactions. They also can be used as shared
accounts. For example, membership fees are not directly associated with exchange transactions.
Intermediate accounts must not interact directly with other intermediate accounts.

\subsection{External Transactions}

External transactions are not initially essential to a currency and introduce risk as they require
interaction with other currencies. Once several currencesy are sufficiently established and
reliable it may be possible to introduce extenal transactions. The external transaction mechanism
would need a way to resolve coordination problems. One reasonable possibility would be to have
special external transaction accounts which are limited in size, and have a fixed exchange rate that
is the ratio of the price indices of the two currencies. 


% \underline{Exchange Transactions}
% 
% 1. Present to the user a value that is the product of the user's base account value and an single,
% global value that we call the demand index ($D_x$). This product is the value that users generally
% see and is the value in which price agreements are made. This value gradually changes as the demand
% index changes, appearing to users much as a bank account that has gradual increases as an interest
% rate is applied to it. As such, it presents no usability difficulties to the user.
% 
% 3. All exchange transactions must be associated with a quantity and a goods or service category. An
% exchange transaction is only valid if both the seller and buyer confirm the same goods or service
% category. This serves three purposes.
% 
% a. It asserts that the transaction is an exchange transaction, and as such, has no associated
% contract with it beyond the current delivery of the goods or service. By doing this it confirms that
% there are no future legal commitments to any future repayment. This is required to limit the use of
% exchange transactions for the purpose of some other transaction category.
% 
% b. The data can be used, with necessary software mechanism to ensure the privacy of the data, to
% accurately calculate the price index. 
% 
% c. It serves as a record for the seller and buyer to resolve any disputes, i.e. it acts as a
% receipt or record of agreement.
% 
% Beyond the seller and buyer entering the same goods and service category, it is easy for seller and
% buyer to collude and provide incorrect information.
% 
% 4. There are intermediate accounts. These are required because all exchange transactions must be
% associated
% 
% TODO
% 
% \underline{Time Transactions}
% 
% 2. Values that set future payments, in particular for repayments on time transactions, are
% denominated in a root value and the product of the price index ($P_x$). In this way, the purchasing
% power of that value remains absolutely constant, and as such there it present any possibility of
% inflation feedback. All repayments on time transactions must be defined at the time the money is
% borrowed and designated as a time transaction.
% 
% \underline{Contract Transactions}
% 
% Contract transactions are prevented by the requirement that borrowers must make repayments to the
% same party that initially lent the money. There are possible ways that users could subvert these
% requirements which we discuss in a later section.
% 
% \underline{External Transactions}
% 
% The provision of an exchange transaction mechanism will be deferred. The possibility of providing
% exchange transaction functionality with external currencies depends on the design of external
% currencies, in particular the accuracy of their price index, and their control over contract
% transactions. If we consider external transactions between two currencies of the kind we are
% presenting in this paper, then an external exchange rate fixed to the relative price index of the
% two currencies, and restricting transactions to exchange transactions by requiring the recording of
% goods or service cateogry and quantity may be a suitable design. This kind of system would require a
% pair of special aggregate accounts specifically for external transactions to pair up inflow and
% outflow. In this system, the pairing of inflow and outflow would halt if either account became
% empty. Another possibility would be to implement a floating exchange rate. The potential difficulty
% with this design is that it is relatively easy for users to use exchange accounts in lieu or time or
% contract transactions, and so it may be possible for users to engage in high-frequency capital
% interactions across the external boundary despite the disincentive of having no legal resort on
% repayment failure.
% 
% \underline{Usability}
% 
% The controls we plan to implement will have minimal impact on any user who intends to use the
% currency for exchange or time transactions. There is an additional requirement for both seller and
% buyer to agree on a transaction and to record goods category. Indexed units of account have been
% used in Chile for a number of decades without significant usability problems, and the use of indexed
% units of account in a digital currency could potentially further simplify their use.
% 
% One important exception, however, are restrictions on the types of repayment schedules for time
% transactions. If repayment schedules are overly flexible, the repayments could possibly be used,
% given people's ingenuity in using financial mechanisms to enrich themselves, as a substitute for
% contract transactions. The extent to which this would happen in practice is unknown, and so starting
% with relatively strict controls and relaxing restrictions with experience is probably a reasonable
% approach.
% 
% \underline{Using One Transaction Category in Lieu of Another}
% 
% The are various methods that users may potentially use to one category of transaction of another.
% The most serious risk is that users find ways to make contract transactions. The main barrier to
% preventing this is to ensure that repayments can only be made to the initial lender's account. The
% second barrier is to ensure that there is no legal protection to protect creditors from debtor's
% failure to make ``repayments''. These barriers should be sufficient on the condition that there are
% no significant economic incentives for engaging in such activity. In general such users will choose to
% use other currencies, rather than a currency with these barriers.
% 

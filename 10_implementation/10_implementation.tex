\section{Implementation}
\label{section:implementation}

\underline{Abstraction Layer over Accounts}

To implement a currency design that does not constrain 


we need to reconsider the interaction between the micro-level, the economic participants and owners
of accounts, and the macro-level requirements.

The design strategy we take is to define the macro-level requirements and then to search for ways to
control the currency so that the currency is likely to map out to those macro-level requirements.   

We specify the macro-level requirements as

1. Sustained stability.

2. Macro-level equilibriation.

Simple accounts that hold a number with some control of the aggregate value in all accounts is the
digital equivalent of paper money. This design is insufficiently controlled to achieve our two
macro-level requirements.   

To achieve these outcomes we apply a abstraction layer over these accounts. This abstraction layer
does the following:

\underline{Exchange Transactions}

1. Present to the user a value that is the product of the user's base account value and an single,
global value that we call the demand index ($D_x$). This product is the value that users generally
see and is the value in which price agreements are made. This value gradually changes as the demand
index changes, appearing to users much as a bank account that has gradual increases as an interest
rate is applied to it. As such, it presents no usability difficulties to the user.

3. All exchange transactions must be associated with a quantity and a goods or service category. An
exchange transaction is only valid if both the seller and buyer confirm the same goods or service
category. This serves three purposes.

a. It asserts that the transaction is an exchange transaction, and as such, has no associated
contract with it beyond the current delivery of the goods or service. By doing this it confirms that
there are no future legal commitments to any future repayment. This is required to limit the use of
exchange transactions for the purpose of some other transaction category.

b. The data can be used, with necessary software mechanism to ensure the privacy of the data, to
accurately calculate the price index. 

c. It serves as a record for the seller and buyer to resolve any disputes, i.e. it acts as a
receipt or record of agreement.

Beyond the seller and buyer entering the same goods and service category, it is easy for seller and
buyer to collude and provide incorrect information.

4. There are intermediate accounts. These are required because all exchange transactions must be
associated

TODO

\underline{Time Transactions}

2. Values that set future payments, in particular for repayments on time transactions, are
denominated in a root value and the product of the price index ($P_x$). In this way, the purchasing
power of that value remains absolutely constant, and as such there it present any possibility of
inflation feedback. All repayments on time transactions must be defined at the time the money is
borrowed and designated as a time transaction.

\underline{Contract Transactions}

Contract transactions are prevented by the requirement that borrowers must make repayments to the
same party that initially lent the money. There are possible ways that users could subvert these
requirements which we discuss in a later section.

\underline{External Transactions}

The provision of an exchange transaction mechanism will be deferred. The possibility of providing
exchange transaction functionality with external currencies depends on the design of external
currencies, in particular the accuracy of their price index, and their control over contract
transactions. If we consider external transactions between two currencies of the kind we are
presenting in this paper, then an external exchange rate fixed to the relative price index of the
two currencies, and restricting transactions to exchange transactions by requiring the recording of
goods or service cateogry and quantity may be a suitable design. This kind of system would require a
pair of special aggregate accounts specifically for external transactions to pair up inflow and
outflow. In this system, the pairing of inflow and outflow would halt if either account became
empty. Another possibility would be to implement a floating exchange rate. The potential difficulty
with this design is that it is relatively easy for users to use exchange accounts in lieu or time or
contract transactions, and so it may be possible for users to engage in high-frequency capital
interactions across the external boundary despite the disincentive of having no legal resort on
repayment failure.

\underline{Usability}

The controls we plan to implement will have minimal impact on any user who intends to use the
currency for exchange or time transactions. There is an additional requirement for both seller and
buyer to agree on a transaction and to record goods category. Indexed units of account have been
used in Chile for a number of decades without significant usability problems, and the use of indexed
units of account in a digital currency could potentially further simplify their use.

One important exception, however, are restrictions on the types of repayment schedules for time
transactions. If repayment schedules are overly flexible, the repayments could possibly be used,
given people's ingenuity in using financial mechanisms to enrich themselves, as a substitute for
contract transactions. The extent to which this would happen in practice is unknown, and so starting
with relatively strict controls and relaxing restrictions with experience is probably a reasonable
approach.

\underline{Using One Transaction Category in Lieu of Another}

The are various methods that users may potentially use to one category of transaction of another.
The most serious risk is that users find ways to make contract transactions. The main barrier to
preventing this is to ensure that repayments can only be made to the initial lender's account. The
second barrier is to ensure that there is no legal protection to protect creditors from debtor's
failure to make ``repayments''. These barriers should be sufficient on the condition that there are
no significant economic incentives for engaging in such activity. In general such users will choose to
use other currencies, rather than a currency with these barriers.


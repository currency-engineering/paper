\section{Conclusion}
\label{section:conclusion}

We summarize the theoretical components according to the level of confidence in the results.

\underline{The error effect and Fisher lines}

This property is determined by physical properties and the inferences made from these rules are
sufficiently precise and measurable to be subject to falisification.

\underline{Aggregate equilibrium}

The property is an aggregate of people's behaviour that has been observed over long-periods or time with
great consistency and is a plausible outcome of people's general incentives. Our model remains
robust even if we relax this assumption to a large degree. Almost all fields of economic research
are predicated implicitly or otherwise on the idea that this condition must be modified heavily to
explain real-world conditions, and therefore require significant re-appraisal given that these
effects are better explained as a property of currency design.

\underline{Inflation feedback and financial bubbles and crashes}

These processes are best explained as positive feedbacks that, while their degree and timing cannot
be reasonable determined quanititatively with precision, if unchecked will have clear consequences
and these consequences have consistently observed. These positive feedback processes are
fundamentally uncontrollable with possible rapid changes that we cannot easily regulate. A
successful control system will strongly limit the impact of positive feedback processes.

\underline{Market driven changes in aggregate quantities and prices}

We cannot expect to be able to predict theses changes, but the rate of change is not sufficiently
fast that it cannot be controlled or adjusted for through feedback regulation.

We hope that this paper can lead to new directions in research. In all new engineering endeavours,
theory and practice diverge. We can take advantage of the relative easy of building digital
currencies to specific design specifications to make currency engineering into an experimental
science. Given the theoretical foundations presented in this paper there is a reasonable high chance
that we can design and build currencies that will not prevent markets from reaching equilibrium,
that that a stable equilibrium can be maintained indefinitely in response to changing conditions. 
